\documentclass[11pt, letterpaper]{article}

\usepackage[cache=false,outputdir=build]{minted}
\usepackage[utf8]{inputenc}
\usepackage[T1]{fontenc}
\usepackage{lmodern}
\usepackage{graphicx}
\usepackage{longtable}
\usepackage{wrapfig}
\usepackage{rotating}
\usepackage{amsmath}
\usepackage{textcomp}
\usepackage{amssymb}
\usepackage{hyperref}
\usepackage[spanish]{babel}
\usepackage[round]{natbib}
\usepackage{listings} % Paquete para incluir y resaltar código
%%% Prueba de configuración
\usepackage{xcolor}   % Paquete para definir colores
% Definir un color verde menos intenso
\definecolor{mygreen}{rgb}{0.0, 0.5, 0.0}  % Un verde oscuro
% Definir las propiedades del codigo
\lstset{
    language=Prolog,
    basicstyle=\ttfamily,  % Estilo básico (monoespaciado)
    keywordstyle=\color{blue},  % Color de las palabras clave
    commentstyle=\color{mygreen},  % Color de los comentarios
    stringstyle=\color{red},  % Color de las cadenas de texto
    numbers=left,  % Mostrar números de línea a la izquierda
    numberstyle=\tiny\color{gray},  % Estilo de los números de línea
    stepnumber=1,  % Cada línea numerada
    frame=single,  % Añadir marco alrededor del código
    breaklines=true,  % Romper líneas largas automáticamente
    showstringspaces=false  % No mostrar espacios en las cadenas
}

\title{\textsc{Programación Lógica y Funcional} \\
  Presentación del Curso}

\author{Ángel García Báez \\ \emph{ZS24019400@estudiantes.uv.mx} \\ \\
  Maestría en Inteligencia Artificial \\ \\ \textbf{IIIA}
  Instituto de Investigaciones en Inteligencia Artificial \\
  Universidad Veracruzana \\ \emph{Campus Sur, Calle Paseo Lote II,
    Sección 2a, No 112} \\ \emph{Nuevo Xalapa, Xalapa, Ver., México 91097}}

\date{\today}

\begin{document}

\maketitle

\newpage

\section{Algoritmo de unificación.}

Programe en Prolog un algoritmo de unificación. A reportar: i) El algoritmo elegido comentado; ii) Su código, también comentado; iii) Los siguientes ejemplos de la ejecución \textbf{(50 puntos)}: 

\begin{itemize}
    \item $q(Y, g(a, b)), \, p(g(X, X), Y)$.
    \item $r(a, b, c), \, r(X, Y, Z)$.
    \item $\text{mayor}(\text{padre}(Y), Y), \, \text{mayor}(\text{padre}(Z), \text{juan})$.
    \item $\text{conoce}(\text{padre}(X), X), \, \text{conoce}(W, W)$.
\end{itemize}

A continuación se muestra el algoritmo implementado con ayuda de las diapositivas de \cite{guerra2024} y el articulo original de \cite{k89} de donde se escogió el algoritmo de unificación de Robinson al ser el más accesible dentro de mis capacidades. Dicho algoritmo busca la discordancia entre términos y variables, los descompone a modo de comprobar que dichos términos puedan ser unificables, en caso de no serlo, sencillamente no lo hace y se detienen las búsquedas recursivas. En dicha versión del algoritmo se intento implementar una lista que guardara todas las unificaciones en caso de darse, pero lamentablemente no se pudo, en su lugar se dejo  unificar/3, de forma que el tercer elemento sirva como una constante al termino de la unificación que indique si la cosa unifica satisfactoriamente o si por el contrario, en alguno de los casos falla la unificación y no devuelve más que un mensaje indicando dicho fallo.



\begin{lstlisting}
%%%% 1.- UNIFICACION A MANO %%%%
% 1. Caso base: si los dos terminos son iguales, no hay mas sustituciones necesarias.
unificar(X, X, unifica) :- !.
% 2. Unificacion de variables: si el primer termino es una variable.
unificar(Var, Term, [(Var / Term)]) :-
    var(Var), % Verificar que el primer termino es una variable
    \+ Var == Term,  % La variable no debe ser identica al segundo termino
    !. % Corta para que no haga backtracking si esta regla se cumple
% 3. Unificacion de variables: si el segundo termino es una variable.
unificar(Term, Var, [(Var / Term)]) :-
    var(Var), % Verifica que el segundo termino sea una variable
    \+ Var == Term, % Verifica que el segundo termino no es igual al primero
    !. % Evita que haga el backtracking si esta regla se cumple
% 4. Unificacion de terminos compuestos: descomponer y unificar recursivamente.
unificar(f(T1, T2), f(S1, S2), _) :-
    unificar(T1, S1, _), % Unificacion recursiva del primer componente.
    unificar(T2, S2, _). % Unificacion recursiva del segundo componente.
% 5. Unificacion falla si los terminos no pueden unificarse.
unificar(_, _, nounifica).
% unificar(q(Y, g(a, b)), p(g(X, X),Y ),L).
% unificar(r(a, b, c), r(X,Y, Z),L).
% unificar(mayor(padre(Y ),Y ), mayor(padre(Z), juan),L).
% unificar(conoce(padre(X), X), conoce(W,W ),L).
\end{lstlisting}

A continuación se muestran los resultados de las consultas propuestas por el enunciado y añadí una al final para mostrar lo que ocurre cuando la unificación ocurre para un termino pero no para todos los presentes en la consulta:

\begin{lstlisting}
?- unificar(q(Y, g(a, b)), p(g(X, X),Y ),L).
L = nounifica
?- unificar(r(a, b, c), r(X,Y, Z),L).
L = unifica,
X = a,
Y = b,
Z = c
?- unificar(mayor(padre(Y ),Y ), mayor(padre(Z), juan),L).
L = unifica,
Y = Z, Z = juan
?- unificar(conoce(padre(X), X), conoce(W,W ),L).
L = unifica,
W = X, X = padre(X)
?- unificar(f(X, b), f(a, e), R).
X = a
R = nounifica
\end{lstlisting}

Dentro de los resultados obtenidos, se muestra que la primer consulta no unifica, las siguientes 3 consultas si unifican completamente y que la consulta propuesta al final unifica solo 1 termino, pero al no poder unificar el otro termino, el algoritmo se detiene y devuelve que no es posible unificar toda la expresión.
Podría omitirse el uso de la variable que reporta si la cosa unifica o no convirtiéndola en una variable anónima  pero decidió dejarse por cuestiones de que sea didáctico, y me sirve para corroborar si la cosa logra su cometido satisfactoriamente.


\newpage

\section{Programa para pino/1.}

Escriban un predicado pino/1 cuyo argumento es un entero positivo y su salida es como sigue \textbf{(10 puntos)}:

\begin{lstlisting}
?- pino(5).
    * 
   * *
  * * *
 * * * *
* * * * *
true.
\end{lstlisting}

A continuación se muestra el código resultante para poder crear el predicado pino/1 el cual al recibir un entero positivo da la salida de un arbolista bonito y centrado.

\begin{lstlisting}
%%%% 2.- Dibujar el pino/1 %%%%
%%% Dibuja un pino de altura N %%%
pino(N) :-
    pino(N, 1).  % Llamada auxiliar con la fila inicial en 1.
% Caso base: Si la fila es mayor que N, termina.
pino(N, F) :-
    F > N, !.  % Corte para evitar backtracking.
% Imprime los espacios en blanco.
pino(N, F) :-
    F =< N,
    Espacios is N - F,  % Calcula el número de espacios en blanco.
    print_espacios(Espacios),  % Imprime los espacios.
    print_asteriscos(F),  % Imprime los asteriscos.
    nl,  % Nueva línea.
    F1 is F + 1,  % Incrementa el contador de filas.
    pino(N, F1).  % Llama recursivamente para la siguiente fila.
% Predicado para imprimir los espacios.
print_espacios(0) :- !.  % Si no hay espacios, no imprime nada.

print_espacios(N) :-
    write(' '),  % Imprime un espacio.
    N1 is N - 1,
    print_espacios(N1).  % Llama recursivamente.
% Predicado para imprimir los asteriscos.
print_asteriscos(0) :- !.  % Si no hay asteriscos, no imprime nada.
print_asteriscos(N) :-
    write('*'),  % Imprime un asterisco.
    write(' '),  % Imprime un espacio después del asterisco.
    N1 is N - 1,
    print_asteriscos(N1).  % Llama recursivamente.
\end{lstlisting}

Después de implementar el código, ocurría un error interesante y es que la lógica que había implementado era correcta dentro de lo que entendía factible para dibujar el árbol centrado, pero al estar programandolo en la versión en la nube de SWI-prolog, siempre me daba el pino pegado a la izquierda, es decir, omitía los espacios para que estuviera centrado. No fue hasta que probé el código en VSC cuando este mismo código, conseguía generar el pino satisfactoriamente como en el ejemplo del enunciado.

Por lo demás, el código funciona y consigue satisfacer adecuadamente lo solicitado en el enunciado.


\newpage

\section{Operaciones de conjuntos.}

Sin usar sus definiciones predefinidas, implemente las siguientes operaciones sobre conjuntos representados como listas \textbf{(20 puntos)}:


\begin{itemize}
    \item Subconjunto:
    \begin{lstlisting}
    ?- subset([1,3],[1,2,3,4]).
    true.
    ?- subset([],[1,2]).
    true.
    \end{lstlisting}
    \item Intersección:
    \begin{lstlisting}
    ?- inter([1,2,3],[2,3,4],L).
    L = [2, 3].
    \end{lstlisting}
    \item Unión:
    \begin{lstlisting}
    ?- union([1,2,3,4],[2,3,4,5],L).
    L = [1, 2, 3, 4, 5].
    \end{lstlisting}
    \item Diferencia:
    \begin{lstlisting}
    ?- dif([1,2,3,4],[2,3,4,5],L).
    L = [1].
    ?- dif([1,2,3],[1,4,5],L).
    L = [2, 3].
    \end{lstlisting}
\end{itemize}


A continuación se muestran las definiciones realizadas de las implementaciones de las operaciones de conjuntos, con ayuda de las encontradas en el libro \cite{clock2003}, las cuales quedan del siguiente modo:

\begin{lstlisting}
%%%% 3.- CONJUNTOS %%%%
%%% Miembro %%%%
% Es necesario definir miembro para reutilizarlo en las siguientes operaciones
miembro(X,[X|_]). % Caso base, X es miembro de la lista de X pegada con la lista vacia
miembro(X,[_|Y]):- miembro(X,Y). % Verificar si X no esta en la cabeza, buscarlo en el resto de la lista
%%% Subconjunto %%%
% Definir que una lista es un subconjunto de otra
subconjunto([],_). % la lista vacia es un subconjunto de la lista Y o _.
subconjunto([A|X],Y):- %Verificar si en la lista [A|X] es subconjunto de Y
    miembro(A,Y), % Verifica si la cabeza de la lista es miembro de la lista Y
    subconjunto(X,Y). % Caso recursivo hasta llegar al caso base
% Esta función solo dice si algo es subconjunto de otro si o no.
%%% Intersección %%%
% Caso base
interseccion([],_,[] ). %La intersección entre la lista vacia y un conjunto X es la lista vacia
interseccion([X|R],Y,[X|Z]):- % Si el primer elemento de la forma parte de Y, entonces X forma parte de la intersección
    miembro(X,Y), % Verifica que X sea miembro en Y
    !, % Si no es miembro, para evitar el backtrackig
    interseccion(R,Y,Z). % Si es miembro aplica la llamada recursiva
interseccion([_|R],Y,Z):- interseccion(R,Y,Z). %Llamada recursiva al resto de la lista hasta agotarla
%%% Union %%% 
% Caso base
union([],X,X). % La union de la lista vacia con X es X
% Condicion de recursividad
union([X|R],Y,Z):-  % Si la cabeza de la lista es miembro de Y, no se añade a la lista para evitar que se dupliquen.
    miembro(X,Y), % Verifica que X sea miembro de la lista Y
    !, % Si es cierto, no hace backtracking,
    union(R,Y,Z). % Si lo es, dispara la recursividad sobre el resto de la lista
union([X|R],Y,[X|Z]):- union(R,Y,Z). % Si el primero no es miembro, continua buscando sobre el resto de la lista R.
%%% Diferencia %%%
% Caso base: La diferencia de una lista vacía con cualquier lista es una lista vacía.
diferencia([], _, []). 
% Caso recursivo: Si el elemento X de la primera lista no está en la segunda lista,
% se incluye en el resultado de la diferencia.
diferencia([X | R], Y, [X | Z]) :-
    \+ miembro(X, Y),  % Verifica que X no sea miembro de Y.
    diferencia(R, Y, Z).  % Llama recursivamente para el resto de la lista.
% Caso recursivo: Si el elemento X está en la segunda lista, se ignora y se continúa.
diferencia([X | R], Y, Z) :-
    miembro(X, Y),  % Verifica que X sea miembro de Y.
    diferencia(R, Y, Z).  % Llama recursivamente para el resto de la lista.
% Consultas
%subconjunto([1,3],[1,2,3,4]).
%subconjunto([],[1,2]).
%interseccion([1,2,3],[2,3,4],L).
%union([1,2,3,4],[2,3,4,5],L).
%diferencia([1,2,3,4],[2,3,4,5],L).
%diferencia([1,2,3],[1,4,5],L).
\end{lstlisting}


Se probaron todas las metas que sugiere el enunciado sobre las definiciones propuestas anteriormente y al realizar las querys, el resultado fue que todas las querys (con los nombres un poco cambiados para los enunciados) llegan a los mismos resultados de forma satisfactoria.

\newpage

\section{Permutaciones.}

Escriba un programa que regrese en su segundo argumento la lista de todas las permutaciones de la lista que es su primer argumento. Esto sin usar permutation/2 definida por Prolog. (10 puntos) Por ejemplo:

\begin{lstlisting}
?- perms([1,2,3],L).
L = [[1, 2, 3], [1, 3, 2], [2, 1, 3], [2, 3, 1], 
    [3, 1, 2], [3, 2, 1]].
\end{lstlisting}

Para poder implementar dicho algoritmo, se tomo de base el algoritmo encontrado en el libro de \cite{clock2003} al cual se le hizo una pequeña modificación para poder obtener todas las permutaciones en una lista de listas a traves de usar findall. La implementación es la siguiente:

\begin{lstlisting}
%%%% 4.- Permutación de listas %%%%
% Caso base: La permutación de una lista vacía es la lista vacía.
permutacion([], []).
% Caso recursivo: Selecciona un elemento y permuta el resto.
permutacion(L, [H | T]) :-
    append(V, [H | U], L),  % Descompone la lista L en cabeza H y cola U.
    append(V, U, Resto),    % Crea la lista con el resto sin la cabeza H
    permutacion(Resto, T).  % Llamada recursiva sobre el resto.
% Sacar todas las permutaciones en una sola lista usando la definición de permutación
permutaciones(L, Permutaciones) :-
    findall(P, permutacion(L, P), Permutaciones).
% consulta
% permutaciones([1, 2, 3], L).
\end{lstlisting}

Al correr la consulta se logra llegar satisfactoriamente al mismo resultado planteado en el enunciado tal que:

\begin{lstlisting}
?- permutaciones([1, 2, 3], L).
L = [[1,2,3],[1,3,2],[2,1,3],[2,3,1],
    [3,1,2],[3,2,1]]
\end{lstlisting}




\newpage

\section{Números de Peano.}

Escriba un predicado que convierta números naturales de Peano (Recuerden: s(s(s(0))) = 3) a su equivalente decimal. Posteriormente implemente la suma y la resta entre dos números de Peano (10 puntos). Por ejemplo:

\begin{lstlisting}
?- peanoToNat(s(s(s(0))),N).
N = 3.
?- peanoToNat(0,N).
N = 0.
?- sumaPeano(s(s(0)),s(0),R).
R = s(s(s(0))).
?- restaPeano(s(s(0)),s(0),R).
R = s(0).
\end{lstlisting} 

A continuación se muestra el programa escrito en prolog para implementar números naturales de Peano a su equivalencia decimal, la suma y la resta entre estos elementos, para implementarlo fue necesario estudiar primero que son los numeros de Peano y para ello me base en el articulo de \cite{peano2019} donde da una explicación sobre los axiomas y el como se usan:

\begin{lstlisting}
%%%%% 5.- NUMEROS DE PEANO %%%%
% 1. Conversión de un número en notación de Peano a su equivalente decimal.
% Caso base: el equivalente decimal de 0 es 0.
peanoToNat(0, 0).
% Caso recursivo: para s(P), el valor es N + 1.
peanoToNat(s(P), N) :-
    peanoToNat(P, N1),
    N is N1 + 1.
% 2. Suma de dos números de Peano.
% Caso base: sumar 0 con cualquier número da ese mismo número.
sumaPeano(0, Y, Y).
% Caso recursivo: sumar s(X) con Y es equivalente a s(X + Y).
sumaPeano(s(X), Y, s(R)) :-
    sumaPeano(X, Y, R).
% 3. Resta de dos números de Peano.
% Caso base: restar 0 de cualquier número da ese mismo número.
restaPeano(X, 0, X).

% Caso recursivo: restar s(Y) de s(X) es equivalente a restar Y de X.
restaPeano(s(X), s(Y), R) :-
    restaPeano(X, Y, R).
% Metas que se computaron:
%peanoToNat(s(s(s(0))),N).
%N = 3.
%peanoToNat(0,N).
%N = 0.
%sumaPeano(s(s(0)),s(0),R).
%R = s(s(s(0))).
%restaPeano(s(s(0)),s(0),R).
%R = s(0).
\end{lstlisting} 

Tras la implementación del código, se computaron las metas propuestas por el enunciado y se llegaron satisfactoriamente a los mismos resultados. Se incluye el código anexo con el documento.




\newpage


\bibliographystyle{apalike}
\bibliography{Bibliográfia}

\end{document}
